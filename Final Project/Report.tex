\documentclass[11pt,a4paper]{article}

\usepackage[applemac]{inputenc}
\usepackage{latexsym}
\usepackage{graphics}
\usepackage[english]{babel}


\usepackage{amsmath,amssymb}
\usepackage{pstricks,pst-plot}
\usepackage{calc}
\usepackage{multicol}
\usepackage{fancyhdr}
\usepackage{lastpage}
\usepackage[T1]{fontenc}
\usepackage{stmaryrd}
\pagestyle{plain}



\begin{document}
	\title{Optimistic approaches in Contextual Linear Bandit models}
	\author{Mathurin \textsc{Massias} \and Cl�ment \textsc{Nicolle}}
	\date{\today} 
	\maketitle
	\tableofcontents
\hspace{-6mm}

\newpage
\section{Introduction}

%TODO What is a contextual bandit problem ? \\Give examples
\section{Presentation of LinUCB}

\subsection{Principle}

At each iteration we compute an estimate of $\theta$ using a linear regression: since for all $s$ in $\llbracket 1, t \rrbracket$, we have $x_s^{\top} \theta = r_s$, left-multiplying each side by $x_s$ and summing for all indices from $1$ to $t$, we have 
%
$$(\sum\limits_{1}^{t}  x_s x_s^{\top} ) \theta =   \sum\limits_{1}^{t} r_s x_s$$
$$\mathrm{i.e. \,}A \theta = b$$
Using this formula, even if the number of iterations grows, $A$ remains of fixed size. At each iteration we can refine our observation for $\theta$.
\\Initializing $A$ to $I_d$ corresponds to a ridge regression : indeed, if we denote $D_t$ the $t \times d$ design matrix at iteration $t$ : $D_t = \begin{pmatrix} x_1^{\top} \\ ... \\x_t^{\top} \end{pmatrix}$, we have 
$$\theta = (D_t^{\top} D_t + I_d)^{-1} D_t R_t$$
which is the formula for a least square ridge regression with regularization parameter equal to 1.
%TODO influence of this parameter ?
\subsection{Impact of exploration parameter}


\section{Comparison with contextual Thompson sampling}


\section{Analysis of OFUL algorithm}



\end{document}