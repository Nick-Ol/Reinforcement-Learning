\documentclass[a4paper, 12pt]{article}
 
\usepackage[applemac]{inputenc}
\usepackage{graphicx}
\usepackage[french]{babel}
\usepackage[T1]{fontenc}
\usepackage{lmodern}
\usepackage{float}
\usepackage{fullpage}


 
\begin{document}
 
\title{RL : Programming Assignment 4\\Mountain Car}
\author{Mathurin \textsc{Massias} \and Clement \textsc{Nicolle}}
\date{\today} 
 
\maketitle

\section{Position of the problem}

To simulate the inelastic wall, we say that if $x(t+1) = -1.2$ then we calculate $v(t+1)$ with $v(t)$ set to zero. This way, the negative speed is canceled and the car can move to the right with positive speed.\\

Intuitively, the optimal value function would reach a maximum at position $x = 0.6$. On positive positions, it would be huge for high speeds, and lower for lower speeds as it won't be enough to reach the top of the mountain. It would also be large for negative positions and speeds, as it is then gaining momentum before to climb the mountain.

\section{Approximate Value Iteration : Fitted-Q iteration} 

At each n draws of couples $(s,a)$, we learn the parameters $\alpha$ with a least-squares regression. In order to gain time, we initialize the $\alpha$ with negative values, as the value function should be negative (reward = 0 in position 0,6 and -1 elsewhere).\\
In the deterministic case ($\epsilon = 0$), with 20 features, and 5000 draws at each iteration, we get a convergence in 5 iterations, and here is what we obtain for the approximated optimal value function :

\begin{figure}[H]
	\centering
	\noindent\includegraphics[scale=0.3]{fittedQ-5ep-5000draws-deterministic.png}
	\noindent\includegraphics[scale=0.3]{fittedQ-5ep-5000draws-deterministic-levels.png}
	\caption{Left: Approximated optimal value function with fitted-Q for 20 features, 5000 draws 100 times, in the deterministic case, in 3D - Right : Level values}
\end{figure}

As expected, we see that the approximation is less precise.

If we take only 10 features, we get :

\begin{figure}[H]
	\centering
	\noindent\includegraphics[scale=0.3]{fittedQ-10feat-determ.png}
	\noindent\includegraphics[scale=0.3]{fittedQ-10feat-determ-flat.png}
	\caption{Left: Approximated optimal value function with fitted-Q for 10 features, in the deterministic case, in 3D - Right : Level values}
\end{figure}

Whereas when increasing to 30 features, we get :
\begin{figure}[H]
	\centering
	\noindent\includegraphics[scale=0.3]{fittedQ-30feat-determ.png}
	\noindent\includegraphics[scale=0.3]{fittedQ-30feat-determ-flat.png}
	\caption{Left: Approximated optimal value function with fitted-Q for 30 features, in the deterministic case, in 3D - Right : Level values}
\end{figure}

And in the deterministic case, with 20 features, we obtain :
\begin{figure}[H]
	\centering
	\noindent\includegraphics[scale=0.3]{fittedQ-5ep-5000draws-stochastic.png}
	\noindent\includegraphics[scale=0.3]{fittedQ-5ep-5000draws-stochastic-levels.png}
	\caption{Left: Approximated optimal value function with fitted-Q for 20 features, in the stochastic case, in 3D - Right : Level values}
\end{figure}

In general, we see high values in the right side where we have reached the terminal state, at the middle with high speed in one or another direction which means that we are going to gain enough momentum, and at the top of the opposite mountain on the left.

\section{Approximate Policy Iteration : LSTD}

\end{document} 